\chapterimage{20200713-155238-PTH8411}
\chapter{Scheduling, Interrupts, and Synchronization}

In this chapter we take a closer look at the sheduling services, interrupts and its synchronization. In Zephyr ththreads are used to execute independent tasks of an application. Plase have a look at the related \href{https://docs.zephyrproject.org/latest/reference/kernel/index.html#scheduling-interrupts-and-synchronization}{Section} of the \Zephyr Docu. 


\section{Threads}
Let's take a closer look at threads to start into the sheduling topic. In Zephyr threads are used to execute independent tasks of an application which should not be performed by an ISR. 

Each Thread has a dedicated stack area. This must be initialized in advance. There a basicly 2 ways to declare and spawning a thread. You can define it and run it out of a funciton or alternatively, it can be declared at compile time. 
Plase have a look at the \href{https://docs.zephyrproject.org/latest/reference/kernel/threads/index.html}{Threads Section }of the \Zephyr Docu. 
Enough theory, we learn the rest from the exercise: Learning by doing.


\subsection{Hello World Threading}\index{Hello World Threading}

\begin{exercise}
Write an Application with a thread which makes a LED blink every second. Compile it and flash it to your microprocessor board. 
\end{exercise}

\subsubsection{Solution}

The thread configuration needs a stack size, priority and some variables. This can be done as shown in the following code fragment. 

\inputminted[firstline=30, lastline=36, linenos, bgcolor=lightgray, fontsize=\scriptsize]{c}{../samples/helloWorldThreadingStep0/src/main.c}

The Thread can do some init work at first and then do its thask work in a endless loop. The next code snippet shows a typical thread. 

\inputminted[firstline=39, lastline=62, linenos, bgcolor=lightgray, fontsize=\scriptsize]{c}{../samples/helloWorldThreadingStep0/src/main.c}

Last but not least the task is initialized and launched. The complete example is stored in \mintinline{bash}{./samples/helloWorldThreadingStep0}. Compile it and have fun!.

\inputminted[firstline=65, lastline=75, linenos, bgcolor=lightgray, fontsize=\scriptsize]{c}{../samples/helloWorldThreadingStep0/src/main.c}. 
