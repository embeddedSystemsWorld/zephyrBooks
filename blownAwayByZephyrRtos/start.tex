
\chapterimage{20200127-132101-PTH7292}
\chapter{Getting Ready}

This chapter shows you the first steps with \Zephyr RTOS. At the end you will have a working installation of the development environment as well as compiled a sample application and flashed this binary into the target system. The final target is a flashing LED on your board of choice. Get your feet on the ground!

\section{Getting Started with \Zephyr}\index{Getting Started with \Zephyr}


To install the \Zephyr environment, follow the \href{https://docs.zephyrproject.org/latest/getting_started/index.html}{Getting Started Guide}. It will guide you through the whole installation process of the \Zephyr source code as well the software development kit (SDK) (compiler, linker, etc.)
This will copy and install quite a bunch of files, code and tools onto your workstation. The installation will work on the following operating systems (for details have a look into the hints corner: \ref{hint0}):
\begin{itemize}
    \item Ubuntu 22.04 LTS (the recommended one;-)
    \item Ubuntu 18.04 LTS and later
    \item Max OS/X
    \item Windows 10
\end{itemize}

For the first installation, it is recommended to use the proposed folders and tool versions. Once the installation is up and running, you can tweak it to your needs. 
For example, to install several versions of zephyr on the same development machine, you can clone the installation in separate directories. 
It is also possible to isolate the installation from each other by using docker containers. 
% 
% If you are bound to a (windows) company network, see the hints in the appendix. 
% 
A few coffees later, the environment is installed and ready for a little exercise and a basic build check. 

For productive firmware development, it is usually a good parctice to focus on a specific release rather than working with the development branch. This can be achieved with a simple command line option while running \mintinline[breaklines]{bash}{west init}. Use the follwing \mintinline[breaklines]{bash}{west} command instead of that one mentioed in the getting started recepie: 
\mintinline[breaklines]{bash}{west init --mr v3.3.0 ~/zephyrproject}. This will install the \Zephyr major release 3.3.0 and not the actual development branch. 


\section{Exercises}\index{Exercises}

\subsection{Build and flash your first application}\index{Build and flash your first application}

\begin{exercise}
Compile and build the \emph{blinky} sample for your board/machine of choise and flash it to your hardware. 
\end{exercise}



For this task you will need a hardware. The most simple way is to use a board supported by zephyr. 
A list of all supported boards can be found in the \href{https://docs.zephyrproject.org/latest/boards/index.html}{Supported Boards} section of the \Zephyr documentation. Choose your favourite hardware and get ready. We will be using an ST Nucleo L476RG board for this exercise simply because I have one in my collection  of evaluation kits. 


To build the sample application follow the recepie:

First jump into the zephyr directory of your installation. Do not forget to activate the python environment as described in the getting started guide. 
\begin{bashbox}
source ~/zephyrproject/.venv/bin/activate
cd ~/zephyrproject/zephyr
\end{bashbox}


\begin{bashbox}
west build -p always -b <your-board-name> samples/basic/blinky
\end{bashbox}

Change \mintinline{bash}{<your-board-name>} appropriately for your board. 
For the ST Nucleo L476RG Board we use \mintinline{bash}{nucleo_l476rg}.


\begin{bashbox}
west build -p always -b nucleo_l476rg samples/basic/blinky
\end{bashbox}


Some seconds later and if everything goes well, the following output will show up in your shell console:

\begin{textbox}
...
[157/157] Linking C executable zephyr/zephyr.elf
Memory region         Used Size  Region Size  %age Used
           FLASH:       14110 B       512 KB      2.69%
             RAM:        4024 B       144 KB      2.73%
        IDT_LIST:          0 GB         2 KB      0.00%
\end{textbox}


Connect your target with the workstation and bring it into bootloader mode. Then flash the sample application to the microcontroller by using the \mintinline{bash}{west flash} command:

\begin{bashbox}
west flash
\end{bashbox}

As soon as the targed is flashed the microcontroller will start the blinky application and an LED will start blinking. Your installation ins now up and running. Have fun!


% 
% Now connect the serial line with your workstation e.g. by using an USB to serial converter and open the line with your favorite terminal program (serial line parameters: 115200 8N1). My favorite terminal program is \mintinline{bash}{minicom}. On a windows box \mintinline{bash}{putty} will perfectly do the job. If really everything went well, the following Hello World message will apear in your terminal:
% \begin{bashbox}
% minicom -D /dev/ttyACM0
% \end{bashbox}
% 
% 
% \begin{textbox}
% *** Booting Zephyr OS build zephyr-v2.3.0-2329-g9ac8dcf5a785  ***
% Hello World! nucleo_l476rg
% \end{textbox}

% 

\subsubsection{How-To get the hello world sample running on a SAMD21-board}

To build the sample application use the following commands:

\begin{bashbox}
cd ~/zephyrproject/zephyr
west build -p auto -b <your-board-name> samples/basic/hello_world
\end{bashbox}

Change \mintinline{bash}{<your-board-name>} appropriately for your board. For an Adafruit ItsyBitsy M4 Express Board use \mintinline{bash}{adafruit_itsybitsy_m4_express}.


\begin{bashbox}
west build -p auto -b adafruit_itsybitsy_m4_express samples/basic/hello_world
\end{bashbox}


Some seconds later and if everything goes well, the following output will show up in your shell console:

\begin{textbox}
[114/119] Linking C executable zephyr/zephyr_prebuilt.elf
Memory region         Used Size  Region Size  %age Used
           FLASH:       10404 B       232 KB      4.38%
            SRAM:        3952 B        32 KB     12.06%
        IDT_LIST:          72 B         2 KB      3.52%
[119/119] Linking C executable zephyr/zephyr.elf
\end{textbox}


Connect your target with the workstation and bring it into bootloader mode. Then flash the sample application to the microcontroller by using the \mintinline{bash}{west flash} command:

\begin{bashbox}
west flash
\end{bashbox}

Some seconds later and if everything goes well, the following output will show up in your shell console:

\begin{textbox}
-- west flash: using runner bossac
Atmel SMART device 0x1001000a found
Erase flash
done in 0.825 seconds

Write 10404 bytes to flash (163 pages)
[==============================] 100% (163/163 pages)
done in 0.057 seconds

Verify 10404 bytes of flash with checksum.
Verify successful
done in 0.017 seconds
Set boot flash true
Ignoring set boot from flash flag.
CPU reset.
\end{textbox}

Now connect the serial line with your workstation e.g. by using an USB to serial converter and open the line with your favorite terminal program (serial line parameters: 115200 8N1). My favorite terminal program is \mintinline{bash}{minicom}. On a windows box \mintinline{bash}{putty} will perfectly do the job. If really everything went well, the following Hello World message will apear in your terminal:
\begin{bashbox}
minicom -D /dev/ttyUSB0
\end{bashbox}


\begin{textbox}
*** Booting Zephyr OS build zephyr-v2.3.0-2329-g9ac8dcf5a785  ***                                                                                         
Hello World! adafruit_itsybitsy_m4_express
\end{textbox}



% \subsection{Hello World LED}\index{Hello World LED}
% 
% \begin{exercise}
% Compile and build the Hello World LED  (Blinky) sample for your board/machine and flash it to your hardware. 
% And yes, Lorenz, it is really amazing when a LED flashes, even if it does so on an eval board when you power it for the first time... 
% 
% You will find all samples in the folder  \mintinline[breaklines]{bash}{~/zephyrproject/zephyr/samples/} of your installation. You can try them out as you like and play around until your ears are wiggling (nice german 1:1 translation;-). 
% \end{exercise}

% \begin{exercise}
% Analyze the sources of the examples. In particular the configuration files. 
% \end{exercise}
