\chapterimage{20200702-120319-PTH8248}
\chapter{Facts and Figures about \Zephyr}

\section{\Zephyr in a Glimpse}
\Zephyr is an open source real-time operating system for the small, cute and resource-limited microcontrollers like derivatives from the ARM Cortex M-series. It is a project developed by de Linux Foundation for the Internet of Things. 
Version 1.0.0 was launched in February 2016. Currently (March 2023) version 3.3.0 is available.
The project web site can be found here: \url{https://www.zephyrproject.org/}.

\section{Why shall I use \Zephyr RTOS?}
Well, good question! First of all, its a new and modern RTOS which is security oriented. Furthermore, it has no old legacy. An other advantage for me is, that \Zephyr is not only a plain RTOS kernel. It comes with a solid hardware abstraction layer for most microprocessor pheripherals. OS services like file system handling, logging, debugging, sensor drivers, low power shutdown functionality for battery powered units and crypto handling are also included. On top of that, many IoT protocols like MQTT, CoAP, Thread, BLE, etc. are also part of the game. 
\Zephyr has quite some analogies to linux. For example, the description of the hardware and its interfaces by means of a device tree was adopted from linux. 
This makes it possible to develop applications that are hardware independent to the greatest possible extent. 
So for me it as a much more comprehensive RTOS than many others on the market and therefore well suited for a wide range of microcontroller applications in the IoT field. 

\section{Device Support}

\Zephyr RTOS supports a wide range of CPU Architectures, System on Chip's (SoC) aka Microcontrollers  and HW-Boards. Almost every newer microprocessor and eval boards as well as microprocessor boards from suppliers like adafuit and otherer is supported by \Zephyr and can be used as a base for the first steps of a project or directly for a product. 

\subsection{CPU Architectures}

Currently follwing CPU architectures are supported by \Zephyr RTOS. 
\begin{multicols}{4}
\begin{itemize}
\item arc
\item arm
\item nios2
\item posix
\item riscv
\item x86
\item xtensa
\end{itemize}
\end{multicols}


\subsection{System on Chip}
As of writing the  following ARM SoC Families are supported by \Zephyr RTOS.

\begin{multicols}{4}
\begin{itemize}
\item arm
\item atmel sam
\item atmel sam0
\item bcm vk
\item cypress
\item infineon xmc
\item microchip mec
\item nordic nrf
\item nuvoton
\item nuvoton npcx
\item nxp imx
\item nxp kinetis
\item nxp lpc
\item qemu cortex a53
\item silabs exx32
\item st stm32
\item ti lm3s6965
\item ti simplelink
\item xilinx zynqmp
\end{itemize}
\end{multicols}


Looking into the stm32 and atmel sam0 families folders reveals a huge amount of supported derivatives to the viewer:

\begin{multicols}{4}
\begin{itemize}
\item samd20
\item samd21
\item samd51
\item same51
\item same53
\item same54
\item samr21
\item stm32f0
\item stm32f1
\item stm32f2
\item stm32f3
\item stm32f4
\item stm32f7
\item stm32g0
\item stm32g4
\item stm32h7
\item stm32l0
\item stm32l1
\item stm32l4
\item stm32l5
\item stm32mp1
\item stm32wb
\end{itemize}
\end{multicols}

A full list of all supported SoC's can be found in the repo: \url{https://github.com/zephyrproject-rtos/zephyr/tree/master/soc}.
A list of supported boards can be in the \href{https://docs.zephyrproject.org/latest/boards/index.html}{Supported Boards} section of the \Zephyr documentation. At the end, I'm quite sure, you will find an appropriate microcontroller for your project! If not, try harder ;-).


\section{A word about the Open Source License of \Zephyr}

The Zephyr Project is licensed under Apache 2.0 and can therefore be used for your commercial projects as well to write your own open source application. For details see: \url{https://www.zephyrproject.org/faqs/}.
