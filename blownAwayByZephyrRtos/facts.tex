\chapterimage{20200702-120319-PTH8248}
\chapter{Facts and Figures about \Zephyr}

\section{\Zephyr in a Glimpse}
\Zephyr is an open source real-time operating system for the small, cute and resource-limited microcontrollers like derivatives from the ARM Cortex M-series. It is a project developed by de Linux Foundation for the Internet of Things. 
Version 1.0.0 was launched in February 2016. Currently (April 2022) version 3.0.0 is available.
The project web site can be found here: \url{https://www.zephyrproject.org/}.


\section{Supported devices}

\Zephyr RTOS supports a wide range of CPU Architectures, System on Chip's (SoC) aka Microcontrollers  and HW-Boards. Almost every newer board is supported by \Zephyr and can be used for the first steps. 

\subsection{CPU Architectures}

Currently follwing CPU architectures are supported by \Zephyr RTOS. 
\begin{multicols}{4}
\begin{itemize}
\item arc
\item arm
\item nios2
\item posix
\item riscv
\item x86
\item xtensa
\end{itemize}
\end{multicols}


\subsection{System on Chip}
As of writing the  following ARM SoC Families are supported by \Zephyr RTOS.

\begin{multicols}{4}
\begin{itemize}
\item arm
\item atmel sam
\item atmel sam0
\item bcm vk
\item cypress
\item infineon xmc
\item microchip mec
\item nordic nrf
\item nuvoton
\item nuvoton npcx
\item nxp imx
\item nxp kinetis
\item nxp lpc
\item qemu cortex a53
\item silabs exx32
\item st stm32
\item ti lm3s6965
\item ti simplelink
\item xilinx zynqmp
\end{itemize}
\end{multicols}


Looking into the stm32 and atmel sam0 families folders reveals a huge amount of supported derivatives to the viewer:

\begin{multicols}{4}
\begin{itemize}
\item samd20
\item samd21
\item samd51
\item same51
\item same53
\item same54
\item samr21
\item stm32f0
\item stm32f1
\item stm32f2
\item stm32f3
\item stm32f4
\item stm32f7
\item stm32g0
\item stm32g4
\item stm32h7
\item stm32l0
\item stm32l1
\item stm32l4
\item stm32l5
\item stm32mp1
\item stm32wb
\end{itemize}
\end{multicols}

A full list of all supported SoC's can be found in the repo: \url{https://github.com/zephyrproject-rtos/zephyr/tree/master/soc}.
A list of supported boards can be in the \href{https://docs.zephyrproject.org/latest/boards/index.html}{Supported Boards} section of the \Zephyr documentation. At the end, I'm quite sure, you will find an appropriate microcontroller for your project! If not, try harder ;-).


\section{A word about the Open Source License of \Zephyr}

The Zephyr Project is licensed under Apache 2.0 and can therefore be used for your commercial projects as well to write your own open source application. For details see: \url{https://www.zephyrproject.org/faqs/}.
