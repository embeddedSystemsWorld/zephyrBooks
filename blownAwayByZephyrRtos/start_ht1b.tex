For this task you will need a hardware. The most simple way is to use a board supported by zephyr. 
A list of all supported boards can be found in the \href{https://docs.zephyrproject.org/latest/boards/index.html}{Supported Boards} section of the \Zephyr documentation. Choose your favourite hardware and get ready. We will be using an ST Nucleo L476RG board for this exercise simply because I have one in my collection  of evaluation kits. 


To build the sample application follow the recepie:

First jump into the zephyr directory of your installation. Do not forget to activate the python environment as described in the getting started guide. 
\begin{bashbox}
source ~/zephyrproject/.venv/bin/activate
cd ~/zephyrproject/zephyr
\end{bashbox}


\begin{bashbox}
west build -p always -b <your-board-name> samples/basic/blinky
\end{bashbox}

Change \mintinline{bash}{<your-board-name>} appropriately for your board. 
For the ST Nucleo L476RG Board we use \mintinline{bash}{nucleo_l476rg}.


\begin{bashbox}
west build -p always -b nucleo_l476rg samples/basic/blinky
\end{bashbox}


Some seconds later and if everything goes well, the following output will show up in your shell console:

\begin{textbox}
...
[157/157] Linking C executable zephyr/zephyr.elf
Memory region         Used Size  Region Size  %age Used
           FLASH:       14110 B       512 KB      2.69%
             RAM:        4024 B       144 KB      2.73%
        IDT_LIST:          0 GB         2 KB      0.00%
\end{textbox}


Connect your target with the workstation and bring it into bootloader mode. Then flash the sample application to the microcontroller by using the \mintinline{bash}{west flash} command:

\begin{bashbox}
west flash
\end{bashbox}

As soon as the targed is flashed the microcontroller will start the blinky application and an LED will start blinking. Your installation ins now up and running. Have fun!


% 
% Now connect the serial line with your workstation e.g. by using an USB to serial converter and open the line with your favorite terminal program (serial line parameters: 115200 8N1). My favorite terminal program is \mintinline{bash}{minicom}. On a windows box \mintinline{bash}{putty} will perfectly do the job. If really everything went well, the following Hello World message will apear in your terminal:
% \begin{bashbox}
% minicom -D /dev/ttyACM0
% \end{bashbox}
% 
% 
% \begin{textbox}
% *** Booting Zephyr OS build zephyr-v2.3.0-2329-g9ac8dcf5a785  ***
% Hello World! nucleo_l476rg
% \end{textbox}
